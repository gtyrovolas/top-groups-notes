%!TEX output_directory = .aux
%!TEX copy_output_on_build(true)

\documentclass[11pt,a4paper, titlepage]{article}
\usepackage[a4paper, total={6.5in, 8in}]{geometry}
\usepackage[utf8]{inputenc}
\usepackage{amsfonts}
\usepackage{amssymb}
\usepackage{amsmath}
\usepackage{mathtools}
\usepackage{amsthm}

\title{Topology and Groups Notes}
\author{Giannis Tyrovolas}
\date{December 29, 2020}
\newtheorem{theorem}{Theorem}[section]
\newtheorem{prop}[theorem]{Proposition}
\newtheorem*{remark}{Remark}
\DeclarePairedDelimiter\abs{\lvert}{\rvert}
\DeclarePairedDelimiter\norm{\lVert}{\rVert}

\theoremstyle{definition}
\newtheorem{definition}[theorem]{Definition}
\newtheorem{example}[theorem]{Example}
\newtheorem{corollary}[theorem]{Corollary}
\newtheorem{lemma}[theorem]{Lemma}
\newtheorem{proposition}[theorem]{Proposition}
\newtheorem*{idea}{Idea}

\begin{document}

\maketitle

\section{Simplicial Approximation Theorem}

The simplicial approximation theorem has lots of steps. Which is why I'm going to lay it all out with my comments here.

First, we'll define the star of a point in a simplicial complex.

\begin{definition}[Star]
Let $x \in \abs{K}$ then the star of $x$ in $\abs{K}$ is the union of all open simplices that contain $x$, i.e:

\[
	st_K(x) = \bigcup\{\text{inside of } \sigma | \sigma \text{ is a simplex of } \abs{K}, x \in \sigma\}
\]
\end{definition}

\begin{lemma}
The star of a point is open in the topological realisation of the simplex.
\end{lemma}

\emph{Note:} We can't prove this as a union of open sets as the inside of a simplex need not be open.

Now we will use simplicial maps to induce maps which are homotopic to continuous maps.

\begin{theorem}
Let $f \colon \abs{K} \longrightarrow \abs{L}$ continuous. Suppose that for all $v \in V(K)$ there is a vertext $g(v) \in L$ such that:

\[
	f(st_K(v)) \subseteq st_L(g(v))
\]

Then $g \colon K \longrightarrow L$ is a simplicial map and for a suitable $\abs{g} \colon \abs{K} \longrightarrow \abs{L}$, $\abs{g} \simeq f$
\end{theorem}
 
\begin{proof}
 Let $x \in K$ be inside $\sigma = (v_0, \ldots, v_n)$ with
 \begin{align*}
 &&  x &= \sum \lambda_i v_i  \\
 &\implies& x &\in st_K(v_i) \text{ for each } i = 0,\ldots, n  \\
 &\implies& f(x) &\in st_L(g(v_i)) \text{ by hypothesis} \\ 
 &\implies& f(x) &\in \text{ inside of } (g(v_0), \ldots, g(v_n), \ldots)
 \end{align*}

 but $(g(v_0), \ldots, g(v_n))$ is itself a simplex of $L$. Now define $\abs{g}(x) = \sum \lambda_i g(v_i)$

 We claim that $\abs{g}$ is continuous and homotopic to $f$ by the straight line homotopy.
\end{proof} 

This is the bulk of the theorem, but we need some more ingredients.

\begin{definition}[Standard metric]
The standard metric on a simplicial complex $\abs{K}$ is defined as:
\[
	d(\sum \lambda_i v_i, \sum \mu_i v_i) = \sum \abs{\lambda_i - \mu_i}
\]
\end{definition}

\begin{definition}[Coarsness]
The coarsness of a subdivision $K'$ of $K$ is:
\[
	\sup \{d_K(x,y) \colon x, y \in st_{K'} | v \in V(K')\}
\]
\end{definition}

The coarsness is a measure of the biggest simplex in the subdivision with respect to the distance in the \emph{original} simplex.

We claim that every finite simplicial complex can be subvided in arbitrarily small subvisions.

\begin{lemma}[Legesgue Covering Lemma]
For a compact topological metric space $X$ and any open cover $\mathcal{U}$ there exists a $\delta > 0$ such that any subset with diameter less than $\delta$ is entirely contained in an element of $\mathcal{U}$.
\end{lemma}




\end{document}

